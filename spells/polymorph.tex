\spellentry{Polymorph}
\textbf{School }transmutation (polymorph); \textbf{Level }sorcerer/wizard 5\\
\textbf{Casting Time }1 standard action\\
\textbf{Components }V, S, M (a piece of the creature whose form you choose)\\
\textbf{Range }touch\\
\textbf{Target }living creature touched\\
\textbf{Duration }1 min/level (D)\\
\textbf{Saving Throw }Will negates (harmless); \textbf{Spell Resistance }yes (harmless)\\
This spell transforms a willing creature into an animal, humanoid or elemental of your choosing; the spell has no effect on unwilling creatures, nor can the creature being targeted by this spell influence the new form assumed (apart from conveying its wishes, if any, to you verbally).\\
If you use this spell to cause the target to take on the form of an animal, the spell functions as \textit{beast shape II}. If the form is that of an elemental, the spell functions as \textit{elemental body I. }If the form is that of a humanoid, the spell functions as \textit{alter self}. The subject may choose to resume its normal form as a full-round action; doing so ends the spell for that subject.\\
\textbf{Polymorph, Greater}\\
\textbf{School }transmutation (polymorph); \textbf{Level }sorcerer/wizard 7\\
This spell functions as \textit{polymorph} except that it allows the creature to take on the form of a dragon or plant creature. If you use this spell to cause the target to take on the form of an animal or magical beast, it functions as \textit{beast shape IV}. If the form is that of an elemental, the spell functions as \textit{elemental body III. }If the form is that of a humanoid, the spell functions as \textit{alter self}. If the form is that of a plant, the spell functions as \textit{plant shape II. }If the form is that of a dragon, the spell functions as \textit{form of the dragon I. }The subject may choose to resume its normal form as a full-round action; doing so ends the spell.\\
