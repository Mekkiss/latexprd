\spellentry{Shout}
\textbf{School} evocation [sonic]; \textbf{Level} bard 4, sorcerer/wizard 4\\
\textbf{Casting Time} 1 standard action\\
\textbf{Components} V\\
\textbf{Range} 30 ft.\\
\textbf{Area} cone-shaped burst\\
\textbf{Duration} instantaneous\\
\textbf{Saving Throw} Fortitude partial or Reflex negates (object); see text; \textbf{Spell Resistance} yes (object)\\
You emit an ear-splitting yell that deafens and damages creatures in its path. Any creature within the area is deafened for 2d6 rounds and takes 5d6 points of sonic damage. A successful save negates the deafness and reduces the damage by half. Any exposed brittle or crystalline object or crystalline creature takes 1d6 points of sonic damage per caster level (maximum 15d6). An affected creature is allowed a Fortitude save to reduce the damage by half, and a creature holding fragile objects can negate damage to them with a successful Reflex save. A \textit{shout }spell cannot penetrate a \textit{silence }spell.\\
\textbf{Shout, Greater}\\
\textbf{School} evocation [sonic]; \textbf{Level} bard 6, sorcerer/wizard 8\\
\textbf{Components}: V, S, F (a metal or ivory horn)\\
\textbf{Range} 60 ft.\\
\textbf{Saving Throw} Fortitude partial or Reflex negates (object); see text\\
This spell functions like \textit{shout, }except that the cone deals 10d6 points of sonic damage (or 1d6 points of sonic damage per caster level, maximum 20d6, against exposed brittle or crystalline objects or crystalline creatures). It also causes creatures to be stunned for 1 round and deafened for 4d6 rounds. A creature in the area of the cone can negate the stunning and halve both the damage and the duration of the deafness with a successful Fortitude save. A creature holding vulnerable objects can attempt a Reflex save to negate the damage to those objects.\\
